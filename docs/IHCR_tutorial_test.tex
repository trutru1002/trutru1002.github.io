% Options for packages loaded elsewhere
\PassOptionsToPackage{unicode}{hyperref}
\PassOptionsToPackage{hyphens}{url}
%
\documentclass[
]{book}
\usepackage{amsmath,amssymb}
\usepackage{lmodern}
\usepackage{iftex}
\ifPDFTeX
  \usepackage[T1]{fontenc}
  \usepackage[utf8]{inputenc}
  \usepackage{textcomp} % provide euro and other symbols
\else % if luatex or xetex
  \usepackage{unicode-math}
  \defaultfontfeatures{Scale=MatchLowercase}
  \defaultfontfeatures[\rmfamily]{Ligatures=TeX,Scale=1}
\fi
% Use upquote if available, for straight quotes in verbatim environments
\IfFileExists{upquote.sty}{\usepackage{upquote}}{}
\IfFileExists{microtype.sty}{% use microtype if available
  \usepackage[]{microtype}
  \UseMicrotypeSet[protrusion]{basicmath} % disable protrusion for tt fonts
}{}
\makeatletter
\@ifundefined{KOMAClassName}{% if non-KOMA class
  \IfFileExists{parskip.sty}{%
    \usepackage{parskip}
  }{% else
    \setlength{\parindent}{0pt}
    \setlength{\parskip}{6pt plus 2pt minus 1pt}}
}{% if KOMA class
  \KOMAoptions{parskip=half}}
\makeatother
\usepackage{xcolor}
\usepackage{longtable,booktabs,array}
\usepackage{calc} % for calculating minipage widths
% Correct order of tables after \paragraph or \subparagraph
\usepackage{etoolbox}
\makeatletter
\patchcmd\longtable{\par}{\if@noskipsec\mbox{}\fi\par}{}{}
\makeatother
% Allow footnotes in longtable head/foot
\IfFileExists{footnotehyper.sty}{\usepackage{footnotehyper}}{\usepackage{footnote}}
\makesavenoteenv{longtable}
\usepackage{graphicx}
\makeatletter
\def\maxwidth{\ifdim\Gin@nat@width>\linewidth\linewidth\else\Gin@nat@width\fi}
\def\maxheight{\ifdim\Gin@nat@height>\textheight\textheight\else\Gin@nat@height\fi}
\makeatother
% Scale images if necessary, so that they will not overflow the page
% margins by default, and it is still possible to overwrite the defaults
% using explicit options in \includegraphics[width, height, ...]{}
\setkeys{Gin}{width=\maxwidth,height=\maxheight,keepaspectratio}
% Set default figure placement to htbp
\makeatletter
\def\fps@figure{htbp}
\makeatother
\setlength{\emergencystretch}{3em} % prevent overfull lines
\providecommand{\tightlist}{%
  \setlength{\itemsep}{0pt}\setlength{\parskip}{0pt}}
\setcounter{secnumdepth}{5}
\usepackage{booktabs}
\usepackage{amsthm}
\makeatletter
\def\thm@space@setup{%
  \thm@preskip=8pt plus 2pt minus 4pt
  \thm@postskip=\thm@preskip
}
\makeatother
\ifLuaTeX
  \usepackage{selnolig}  % disable illegal ligatures
\fi
\usepackage[]{natbib}
\bibliographystyle{apalike}
\IfFileExists{bookmark.sty}{\usepackage{bookmark}}{\usepackage{hyperref}}
\IfFileExists{xurl.sty}{\usepackage{xurl}}{} % add URL line breaks if available
\urlstyle{same} % disable monospaced font for URLs
\hypersetup{
  pdftitle={IHCR\_test},
  pdfauthor={Truitt Elliott},
  hidelinks,
  pdfcreator={LaTeX via pandoc}}

\title{IHCR\_test}
\author{Truitt Elliott}
\date{2022-10-19}

\begin{document}
\maketitle

{
\setcounter{tocdepth}{1}
\tableofcontents
}
\hypertarget{opening}{%
\chapter{Opening}\label{opening}}

This is a sample section for what a markdown tutorial might look like using bookdown

\hypertarget{reusable}{%
\chapter{Creating Reusable Data}\label{reusable}}

\hypertarget{reflection}{%
\subsection{\texorpdfstring{\emph{Reflection}}{Reflection}}\label{reflection}}

Have you ever taken over a project from someone else, or had to go through someone else's file system - only to find it to be incomprehensible?~ Have you opened an unknown spreadsheet to see titleless column names, incomprehensible measurements, and miscellaneous notes scattered in additional cells?~ If you can't contact the original creator of the data, how are you supposed to use it?~ These frustrations are common when working with online data, \& can be avoided when publishing your data online - ultimately ensuring that your data stays useful, relevant, \& can be built upon in future projects.

\hypertarget{summary}{%
\subsection{\texorpdfstring{\emph{Summary}}{Summary}}\label{summary}}

If your data/output will be available for open source scrutiny, then the data must be legible to an outsider coming in without consulting you.~ A data dictionary, detailed summary, data connection map, ect - are all tools that should be implemented.~ This allows for future projects to use the data with minimal effort in understanding what they're working with.~~~

\hypertarget{pitfalls}{%
\subsection{\texorpdfstring{\emph{Pitfalls}}{Pitfalls}}\label{pitfalls}}

If your spreadsheet contains columns/rows without identifiers, are connected in ways that aren't documented, or has so many files that it would be impossible for someone not on the project to understand - then it will be useless to any future researchers looking to use the data.~ Your data should always carry supplemental information with explanation, particularly when that data is going to be in long term storage at a project's end.

\hfill\break

  \bibliography{book.bib,packages.bib}

\end{document}
